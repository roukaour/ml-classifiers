\documentclass[11pt]{article}

\def\mainroot{ell_1}

\usepackage[top=1in, bottom=1in, left=1in, right=1in]{geometry}

\usepackage[compact]{titlesec}

\usepackage{xcolor}
\definecolor{pykeyword}{HTML}{000088}
\definecolor{pyidentifier}{HTML}{000000}
\definecolor{pystring}{HTML}{008800}
\definecolor{pynumber}{HTML}{FF4000}
\definecolor{pycomment}{HTML}{880000}

\usepackage{listings}
\lstset{language=Python,
	basicstyle=\ttfamily\small,
	tabsize=4,
	showstringspaces=false,
	keywordstyle=\color{pykeyword},
	identifierstyle=\color{pyidentifier},
	stringstyle=\color{pystring},
	numberstyle=\color{pynumber},
	commentstyle=\color{pycomment}
}

\usepackage{algorithm}
\usepackage{algpseudocode}

\usepackage{graphicx}

\title{CSE 537 Assignment 5 Report: ML Classifiers}

\author{
Remy Oukaour \\
	{\small SBU ID: 107122849}\\
	{\small \texttt{remy.oukaour@gmail.com}}
\and
Jian Yang \\
	{\small SBU ID: 110168771}\\
	{\small \texttt{jian.yang.1@stonybrook.edu}}
}

\date{Wednesday, December 9, 2015}

\raggedbottom

\begin{document}

\maketitle

\section{Introduction}

This report describes our submission for assignment 5 in the CSE 537 course on
artificial intelligence. Assignment 5 requires us to implement two kinds of
machine learning classifiers: a decision tree classifier for predicting credit
worthiness of applicants, and a na{\"i}ve Bayes classifier for classifying
handwritten digits. In this report, we discuss the implementation details and
performance of our solutions.

\section{Decision Tree Classifier}

Jian Yang developed the decision tree classifier for predicting credit worthiness
of applicants.

% TODO

\section{Na{\"i}ve Bayes Classifier}

Remy Oukaour developed the na{\"i}ve Bayes classifier for classifying handwritten
digits.

\begin{figure}[h!]
\centering
\includegraphics[height=2.75in, width=2.75in]{digits.png}
\caption{The 97 digits which the algorithm misclassified.}
\label{nbc_digits}
\end{figure}

\end{document}
