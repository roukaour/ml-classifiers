\documentclass[11pt]{article}

\def\mainroot{ell_1}

\usepackage[top=1in, bottom=1in, left=1in, right=1in]{geometry}

\usepackage[compact]{titlesec}

\usepackage{xcolor}
\definecolor{pykeyword}{HTML}{000088}
\definecolor{pyidentifier}{HTML}{000000}
\definecolor{pystring}{HTML}{008800}
\definecolor{pynumber}{HTML}{FF4000}
\definecolor{pycomment}{HTML}{880000}

\usepackage{listings}
\lstset{language=Python,
	basicstyle=\ttfamily\small,
	tabsize=4,
	showstringspaces=false,
	keywordstyle=\color{pykeyword},
	identifierstyle=\color{pyidentifier},
	stringstyle=\color{pystring},
	numberstyle=\color{pynumber},
	commentstyle=\color{pycomment}
}

\usepackage{algorithm}
\usepackage{algpseudocode}

\usepackage{graphicx}

\usepackage{amsmath}
\usepackage{enumitem}

\title{CSE 537 Assignment 5 Report: ML Classifiers}

\author{
Remy Oukaour \\
	{\small SBU ID: 107122849}\\
	{\small \texttt{remy.oukaour@gmail.com}}
\and
Jian Yang \\
	{\small SBU ID: 110168771}\\
	{\small \texttt{jian.yang.1@stonybrook.edu}}
}

\date{Wednesday, December 11, 2015}

\raggedbottom

\begin{document}

\maketitle

\section{Introduction}

This report describes our submission for assignment 5 in the CSE 537 course on
artificial intelligence. Assignment 5 requires us to implement two kinds of
machine learning classifiers: a decision tree classifier for predicting credit
worthiness of applicants, and a na{\"i}ve Bayes classifier for classifying
handwritten digits. In this report, we discuss the implementation details and
performance of our solutions.

\section{Decision tree classifier}

Jian Yang developed the decision tree classifier for predicting credit worthiness
of applicants.

\subsection{Introduction}
To run the classifier, please run $python\ load.py$. It will load the data in $crx.data.txt$,
and print the result to screen for both the training error and the test error.

\subsection{Algorithm}

To implement the decision tree to classify the data, we used  ID3 algorithm which would pick features
and the split values based on Information Gain defined by:

$$IG(X,Y) = H(Y) - H(Y|X)$$

$$H(Y|X) = \sum_jPr(X=v_j)H(Y|X=v_j)$$

$$H(X) = \sum_{1}^m p_j \log_2{p_j} $$

For each feature, we would calculate the Information Gain for each feature. \\
\subsubsection{Continuous Features}
What is special is for the continuous features, it doesn't make sense to use each value as a single $v_j$.
Instead, we use a binary splitting for continuous features, and enumerate all possible split values, from the smallest value
for this feature to the largest value. \\
For example, if a feature has values $1, 2, 3, 4$, we would try using value one to split it to two sets of data
with the values as $1$ and $2, 3, 4$ and calculate the information gain and then split by $1, 2$ and $3, 4$ and then 
split by $1, 2, 3$ and $4$. By picking the most information gain, we could decide the feature and the related split value.
If in a subset of data the feature has only one value left, we would eliminate the feature in subset of the training data. \\
\subsubsection{Discrete Features}
For discrete features with multiple possible values, we have two choices to build a branch and to calculate the information gain. \\
One is to consider each value has a single branch, which means when we calculate the $H(Y|X)$, we calculate multiple such entropy values for each value. Another solution is using binary branch, which means we could split the value set to two sets and calculate the entropy for the probability an example is in one set. \\
In the first condition, the branch for a discrete feature might have as many branches as the number of values, and for the second condition, there should always be binary branches for the discrete features.


\subsection{Data Set}
The data contains unknown values as $"?"$ for both the continuous features and the discrete features. We use a simple smoothing method,
by choosing the most frequent features of discrete features, and the median of the continuous features. \\
We prepared all the data first to fill values for unknown values $"?"$. \\
And then we shuffled the data and splitted it to training examples and the test examples by half and half. The reason we
need to shuffle first is, the data might be biased for the feature values and the classification labels. \\
For training examples, since we need do post-pruning, we splitted the training data by half and half. And using the first-half
of training data for training and the second-half of training data for testing.

\subsection{Post-Pruning}
After the building of the tree, we used post-pruning to reduce overfitting. The reason of overfitting is there are errors and noises
in data. \\
By post-pruning, we applied the second half training data as validation data to classify based on the tree.
For every node on the tree, we predicted the results, and also calculate the majority of the training data on this node,
if using majority of the training data had higher accuracy than the current node, we would prune this node and its subtrees by using
the majority value directly as the classify rule on this node.

\subsection{Test Result}
Since we randomly split data in every time the script was run, the accuracy is not a fixed number. But in general statistics by running it may times,
we observed that the training error is around 88\% and the test error is around 86\% for multiple branching of discrete features. And the binary branching of discrete features has similar but a little smaller accuracy around 87\% for the training error and 86\% for test error.\\
And the trees with multiple branching of discrete features before post-pruning is showed in fig \ref{before} and after post-pruning is showed in fig \ref{after}. By comparing them we could see
several nodes have been pruned.

\begin{figure}[h!]
\centering
\includegraphics[height=1.18in, width=6.5in]{before.png}
\caption{Decision Tree with multiple branching for discrete features before pruning. ($X[i]$ is feature $A_i$)} 
\label{before}
\end{figure}

\begin{figure}[h!]å
\centering
\includegraphics[height=1.875in, width=6.5in]{after.png}
\caption{Decision Tree with multiple branching for discrete features after pruning. ($X[i]$ is feature $A_i$)}
\label{after}
\end{figure}


As showed in the textbook, we also tested by different sizes of training examples. We used training examples from 10 to 300
and draw the learning curve for test accuracy and the result is showed in fig \ref{learn}.
\begin{figure}[h!]
\centering
\includegraphics[height=3.75in, width=5in]{learn.png}
\caption{Learning curve for test accuracy.}
\label{learn}
\end{figure}

The trees with binary branching of discrete features before post-pruning is showed in fig \ref{bin_before} and after post-pruning is showed in fig \ref{bin_after}.

\begin{figure}[h!]
\centering
\includegraphics[height=4.8in, width=6.5in]{bin_before.png}
\caption{Decision Tree with binary branching for discrete features before pruning. ($X[i]$ is feature $A_i$)}
\label{bin_before}
\end{figure}

\begin{figure}[h!]
\centering
\includegraphics[height=7.62in, width=6.5in]{bin_after.png}
\caption{Decision Tree with binary branching after pruning. ($X[i]$ is feature $A_i$)}
\label{bin_after}
\end{figure}

And if 
\section{Na{\"i}ve Bayes classifier}

Remy Oukaour developed the na{\"i}ve Bayes classifier for classifying handwritten
digits.

\subsection{Instructions}

To run the classifier, enter $python\ naive\text{-}bayes.py$. It will read from
$trainingimages.txt$, $traininglabels.txt$, $testimages.txt$, and $testlabels.txt$,
output to $predictedlabels.txt$ and print a confusion matrix to standard output.

\subsection{Implementation}

The classifier is a straightforward implementation of na{\"i}ve Bayes that supports
arbitrary categorical (a.k.a. generalized Bernoulli or multinoulli) features, using
the formula ``log posterior $\propto$ log prior + log likelihood'':

$$\log P(label|features) \propto \log P(label) + \sum_{feature} P(feature|label)$$

The prior probability $P(label)$ is estimated to be the fraction of training instances
with a given label (0 to 9). The likelihood $P(feature|label)$ of a feature having a
certain value for a test instance is estimated to be the fraction of training instances
(limited to that label) with that value for that feature. (We use Laplace smoothing to
handle novel feature values in the test data, with a smoothing value of 0.001.) We then
pick the label of each test instance using a maximum likelihood estimator:

$$classification(features) = \underset{labels}{\operatorname{argmax}} \ \log P(label|features)$$

\subsection{Feature selection}

We tested many different kinds of features to maximize performance. First, we decided
how to use the images' pixel values: whether to treat gray as black, and whether to use
groups of pixels as individual features. We counted the number of correctly classified
test instances for each alternative.

\begin{itemize}[noitemsep]
\item Individual pixels: 772 correct
\item 2$\times$2 blocks: 849 correct
\item 2$\times$2 overlapping blocks: 861 correct
\item 3$\times$3 blocks, 1-overlapping: 819 correct
\item 3$\times$3 blocks, 2-overlapping: 821 correct
\item 4$\times$4 blocks: 689 correct
\item 4$\times$4 blocks, 1-overlapping: 725 correct
\item 4$\times$4 blocks, 2-overlapping: 742 correct
\item 4$\times$4 blocks, 3-overlapping: 758 correct
\end{itemize}

We chose to use 2$\times$2 overlapping blocks of pixels as features.

We also tried treating gray pixels (``+'') as black ones (``\#''), but this lowered
accuracy from 861 to 857 correct.

The Laplace smoothing value of 0.001 was likewise chosen via testing. We expected 1 to
work well, but it did not improve the accuracy from 861. Higher values actually lowered
the accuracy, so we tried lower values until they no longer provided more benefit. With
smoothing of 0.001, we reached 893 correctly classified test instances.

At this point we added an minimum entropy threshold for the features, on the grounds
that a low-entropy feature would just be adding noise. (Many pixel-block features around
the edges of the digit images are completely white for all training and test images,
thus having 0 entropy and providing no benefit.) Testing a range of threshold values from
0 to 0.3, we found a peak at 0.15, with 897 correct classifications.

To achieve at least 90\% accuracy, we added holistic features:

\begin{itemize}[noitemsep]
\item $num\_regions$: a count of contiguous regions in the image, treating gray and black
pixels as ``foreground'' and white as ``background,'' counted using a flood-fill algorithm.
\item $spread\_ratio$: the ratio of vertical to horizontal foreground ``spread.'' Spread
is the total distance of all foreground pixels from a center line.
\item $horizontal\_bias$: the difference between the number of foreground pixels in the
top and bottom halves of the image.
\item $vertical\_bias$: the difference between the number of foreground pixels in the
left and right halves of the image.
\end{itemize}

By adding these features to the existing block-based ones, we were able to correctly
classify 903 out of 1,000 test instances.

\subsection{Performance results}

The classifier's precision for all ten digits ranged from 79.5\% (for 8) to 96.3\% (for 1).
The sensitivity ranged from 81.1\% (for 7) to 95.6\% (for 0). The overall accuracy was 90.3\%.

\begin{table}[h!]
\centering
\setlength\tabcolsep{4pt}
\begin{tabular}{| r | r | r | r | r | r | r | r | r | r | r | r | r |}
\hline
\textbf{pred\textbackslash true} & \textbf{0} & \textbf{1} & \textbf{2} & \textbf{3} & \textbf{4} & \textbf{5} & \textbf{6} & \textbf{7} & \textbf{8} & \textbf{9} & \textbf{total} & \textbf{precision} \\
\hline
\textbf{0} & 86 & 0 & 1 & 0 & 0 & 1 & 1 & 0 & 1 & 1 & 91 & 94.5\% \\
\hline
\textbf{1} & 0 & 103 & 0 & 0 & 0 & 0 & 1 & 3 & 0 & 0 & 107 & 96.3\% \\
\hline
\textbf{2} & 0 & 1 & 95 & 2 & 0 & 1 & 0 & 5 & 2 & 0 & 106 & 89.6\% \\
\hline
\textbf{3} & 0 & 0 & 1 & 91 & 0 & 2 & 0 & 0 & 7 & 2 & 103 & 88.3\% \\
\hline
\textbf{4} & 0 & 1 & 0 & 0 & 101 & 0 & 0 & 0 & 2 & 3 & 107 & 94.4\% \\
\hline
\textbf{5} & 0 & 0 & 0 & 3 & 0 & 81 & 4 & 1 & 1 & 1 & 91 & 89.0\% \\
\hline
\textbf{6} & 1 & 1 & 1 & 0 & 2 & 0 & 83 & 0 & 0 & 0 & 88 & 94.3\% \\
\hline
\textbf{7} & 1 & 0 & 1 & 1 & 1 & 0 & 0 & 86 & 1 & 1 & 92 & 93.5\% \\
\hline
\textbf{8} & 2 & 2 & 4 & 1 & 0 & 6 & 2 & 2 & 89 & 4 & 112 & 79.5\% \\
\hline
\textbf{9} & 0 & 0 & 0 & 2 & 3 & 1 & 0 & 9 & 0 & 88 & 103 & 85.4\% \\
\hline
\textbf{total} & 90 & 108 & 103 & 100 & 107 & 92 & 91 & 106 & 103 & 100 & 1000 & \\
\hline
\textbf{sensitivity} & 95.6\% & 95.4\% & 92.2\% & 91.0\% & 94.4\% & 88.0\% & 91.2\% & 81.1\% & 86.4\% & 88.0\% & & \\
\hline
\end{tabular}
\caption{Confusion matrix with precision and sensitivity scores for all ten labels.}
\label{nbc_confusion_matix}
\end{table}

The most common errors were to misclassify a 7 as a 9 (which happened 9 times), an 8 as
a 3 (which happened 7 times), or a 5 as an 8 (which happened 6 times). These are all the
kind of errors that a human judge could also make with sloppy handwriting. In addition,
of the 97 misclassified digits, some are illegible even for humans.

\begin{figure}[h!]
\centering
\includegraphics[height=2.75in, width=2.75in]{digits.png}
\caption{The 97 digits which the na{\"i}ve Bayes classifier got wrong.}
\label{nbc_digits}
\end{figure}

\end{document}

